% Options for packages loaded elsewhere
\PassOptionsToPackage{unicode}{hyperref}
\PassOptionsToPackage{hyphens}{url}
\PassOptionsToPackage{dvipsnames,svgnames,x11names}{xcolor}
%
\documentclass[
]{article}

\usepackage{amsmath,amssymb}
\usepackage{iftex}
\ifPDFTeX
  \usepackage[T1]{fontenc}
  \usepackage[utf8]{inputenc}
  \usepackage{textcomp} % provide euro and other symbols
\else % if luatex or xetex
  \usepackage{unicode-math}
  \defaultfontfeatures{Scale=MatchLowercase}
  \defaultfontfeatures[\rmfamily]{Ligatures=TeX,Scale=1}
\fi
\usepackage{lmodern}
\ifPDFTeX\else  
    % xetex/luatex font selection
\fi
% Use upquote if available, for straight quotes in verbatim environments
\IfFileExists{upquote.sty}{\usepackage{upquote}}{}
\IfFileExists{microtype.sty}{% use microtype if available
  \usepackage[]{microtype}
  \UseMicrotypeSet[protrusion]{basicmath} % disable protrusion for tt fonts
}{}
\makeatletter
\@ifundefined{KOMAClassName}{% if non-KOMA class
  \IfFileExists{parskip.sty}{%
    \usepackage{parskip}
  }{% else
    \setlength{\parindent}{0pt}
    \setlength{\parskip}{6pt plus 2pt minus 1pt}}
}{% if KOMA class
  \KOMAoptions{parskip=half}}
\makeatother
\usepackage{xcolor}
\setlength{\emergencystretch}{3em} % prevent overfull lines
\setcounter{secnumdepth}{5}
% Make \paragraph and \subparagraph free-standing
\ifx\paragraph\undefined\else
  \let\oldparagraph\paragraph
  \renewcommand{\paragraph}[1]{\oldparagraph{#1}\mbox{}}
\fi
\ifx\subparagraph\undefined\else
  \let\oldsubparagraph\subparagraph
  \renewcommand{\subparagraph}[1]{\oldsubparagraph{#1}\mbox{}}
\fi


\providecommand{\tightlist}{%
  \setlength{\itemsep}{0pt}\setlength{\parskip}{0pt}}\usepackage{longtable,booktabs,array}
\usepackage{calc} % for calculating minipage widths
% Correct order of tables after \paragraph or \subparagraph
\usepackage{etoolbox}
\makeatletter
\patchcmd\longtable{\par}{\if@noskipsec\mbox{}\fi\par}{}{}
\makeatother
% Allow footnotes in longtable head/foot
\IfFileExists{footnotehyper.sty}{\usepackage{footnotehyper}}{\usepackage{footnote}}
\makesavenoteenv{longtable}
\usepackage{graphicx}
\makeatletter
\def\maxwidth{\ifdim\Gin@nat@width>\linewidth\linewidth\else\Gin@nat@width\fi}
\def\maxheight{\ifdim\Gin@nat@height>\textheight\textheight\else\Gin@nat@height\fi}
\makeatother
% Scale images if necessary, so that they will not overflow the page
% margins by default, and it is still possible to overwrite the defaults
% using explicit options in \includegraphics[width, height, ...]{}
\setkeys{Gin}{width=\maxwidth,height=\maxheight,keepaspectratio}
% Set default figure placement to htbp
\makeatletter
\def\fps@figure{htbp}
\makeatother

\makeatletter
\makeatother
\makeatletter
\makeatother
\makeatletter
\@ifpackageloaded{caption}{}{\usepackage{caption}}
\AtBeginDocument{%
\ifdefined\contentsname
  \renewcommand*\contentsname{Table of contents}
\else
  \newcommand\contentsname{Table of contents}
\fi
\ifdefined\listfigurename
  \renewcommand*\listfigurename{List of Figures}
\else
  \newcommand\listfigurename{List of Figures}
\fi
\ifdefined\listtablename
  \renewcommand*\listtablename{List of Tables}
\else
  \newcommand\listtablename{List of Tables}
\fi
\ifdefined\figurename
  \renewcommand*\figurename{Figure}
\else
  \newcommand\figurename{Figure}
\fi
\ifdefined\tablename
  \renewcommand*\tablename{Table}
\else
  \newcommand\tablename{Table}
\fi
}
\@ifpackageloaded{float}{}{\usepackage{float}}
\floatstyle{ruled}
\@ifundefined{c@chapter}{\newfloat{codelisting}{h}{lop}}{\newfloat{codelisting}{h}{lop}[chapter]}
\floatname{codelisting}{Listing}
\newcommand*\listoflistings{\listof{codelisting}{List of Listings}}
\makeatother
\makeatletter
\@ifpackageloaded{caption}{}{\usepackage{caption}}
\@ifpackageloaded{subcaption}{}{\usepackage{subcaption}}
\makeatother
\makeatletter
\@ifpackageloaded{tcolorbox}{}{\usepackage[skins,breakable]{tcolorbox}}
\makeatother
\makeatletter
\@ifundefined{shadecolor}{\definecolor{shadecolor}{rgb}{.97, .97, .97}}
\makeatother
\makeatletter
\makeatother
\makeatletter
\makeatother
\ifLuaTeX
  \usepackage{selnolig}  % disable illegal ligatures
\fi
\IfFileExists{bookmark.sty}{\usepackage{bookmark}}{\usepackage{hyperref}}
\IfFileExists{xurl.sty}{\usepackage{xurl}}{} % add URL line breaks if available
\urlstyle{same} % disable monospaced font for URLs
\hypersetup{
  pdftitle={Note 2},
  pdfauthor={Hans Martinez},
  colorlinks=true,
  linkcolor={blue},
  filecolor={Maroon},
  citecolor={Blue},
  urlcolor={Blue},
  pdfcreator={LaTeX via pandoc}}

\title{Note 2}
\author{Hans Martinez}
\date{2024-01-16}

\begin{document}
\maketitle
\ifdefined\Shaded\renewenvironment{Shaded}{\begin{tcolorbox}[borderline west={3pt}{0pt}{shadecolor}, interior hidden, enhanced, frame hidden, breakable, sharp corners, boxrule=0pt]}{\end{tcolorbox}}\fi

\hypertarget{perfect-competition-and-productivity}{%
\section{Perfect Competition and
Productivity}\label{perfect-competition-and-productivity}}

Price-taking firms in an industry have the same technology but different
productivity. That is firms produce output \(Q\), with one input \(L\),
given the production function, \(Q=A_i L^{\alpha}\), where \(\alpha<1\)
and \(A_i\ge1\). In other words, their production function is the same
but differs in one parameter, \(A_i\). In this framework, the parameter
\(A_i\) is the productivity of firm \(i\).

Output price is \(P\) and input price is \(w\). The profit-maximization
problem of the firm is

\begin{equation}\protect\hypertarget{eq-prof-max-prob}{}{
\max_{L} \pi(L) = PA_i L^{\alpha}-wL
}\label{eq-prof-max-prob}\end{equation}

Then, the optimality condition yields

\[
\alpha P A_i L^{\alpha-1} = w
\]

Rearranging and solving for \(L\) (remember \(\alpha<1\)), we get

\begin{equation}\protect\hypertarget{eq-input-demand}{}{
L_i^*=\left(\frac{\alpha P A_i}{w}\right)^{\frac{1}{1-\alpha}}
}\label{eq-input-demand}\end{equation}

Equation~\ref{eq-input-demand} is the input demand of firm \(i\). Input
demand is increasing in the output price and productivity, but it is
decreasing in the input price. In other words, the higher the output
price or the productivity, the higher the quantity of inputs the firms
will consume. On the other hand, the higher the input price the less of
the input the firm will consume.

With the input demand at hand, we can find the optimal output quantity
and the profits of the firm.

Plugging Equation~\ref{eq-input-demand} into the production function we
get the optimal output quantity,

\begin{equation}\protect\hypertarget{eq-optimal-output}{}{
Q_i^* = A_i^{\frac{1}{1-\alpha}} \left(\frac{\alpha P}{w}\right)^{\frac{\alpha}{1-\alpha}}
}\label{eq-optimal-output}\end{equation}

Using the optimal output, we can find the profits of the firm by
plugging Equation~\ref{eq-optimal-output} into the profit function,

\begin{equation}\protect\hypertarget{eq-opt-profits}{}{
\pi_i^*=A_i^{\frac{1}{1-\alpha}}\left(\frac{ P}{w^{\alpha}}\right)^{\frac{1}{1-\alpha}}(\alpha^{\frac{\alpha}{1-\alpha}}-\alpha^{\frac{1}{1-\alpha}})
}\label{eq-opt-profits}\end{equation}

Suppose there are only two firms in the industry, but they are still
price takers. So, we are still in a perfectly competitive market, but
firm 1 is more productive than firm 2, \(A_1 >A_2\).

Then, given Equation~\ref{eq-input-demand}, \ref{eq-optimal-output}, and
\ref{eq-opt-profits}, we can conclude that the more productive firm,
firm 1, consumes more inputs \(L_1>L_2\), produces more output
\(Q_1>Q_2\), and has higher profits \(\pi_1>\pi_2\) than firm 2.

Moreover, their market shares are a function of only their
productivities. The market share of firm \(i\) is
\(s_i=\frac{Q_i}{Q_1+Q_2}\). Plugging in
Equation~\ref{eq-optimal-output}, we get

\begin{equation}\protect\hypertarget{eq-mkt-shr}{}{
s_i=\frac{1}{1+\left(\frac{A_j}{A_i}\right)^{\frac{1}{1-\alpha}}}
}\label{eq-mkt-shr}\end{equation}

So, for firm 1 \[
s_1=\frac{1}{1+\left(\frac{A_2}{A_1}\right)^{\frac{1}{1-\alpha}}}
\]

Equation~\ref{eq-mkt-shr} says that the higher the productivity of firm
\(i\), relative to the productivity of the other firm \(j\), the higher
the market share she will get. To see this, note that
\(A_2 < A_1 \implies \frac{A_2}{A_1} < 1\). As this term
\(\frac{A_j}{A_i}\) goes to zero, the market share \(s_i\) gets closer
to 1. Put differently, a firm's market share depends on how much more
productive it is than its competitors.



\end{document}
