% Options for packages loaded elsewhere
\PassOptionsToPackage{unicode}{hyperref}
\PassOptionsToPackage{hyphens}{url}
\PassOptionsToPackage{dvipsnames,svgnames,x11names}{xcolor}
%
\documentclass[
]{article}

\usepackage{amsmath,amssymb}
\usepackage{iftex}
\ifPDFTeX
  \usepackage[T1]{fontenc}
  \usepackage[utf8]{inputenc}
  \usepackage{textcomp} % provide euro and other symbols
\else % if luatex or xetex
  \usepackage{unicode-math}
  \defaultfontfeatures{Scale=MatchLowercase}
  \defaultfontfeatures[\rmfamily]{Ligatures=TeX,Scale=1}
\fi
\usepackage{lmodern}
\ifPDFTeX\else  
    % xetex/luatex font selection
\fi
% Use upquote if available, for straight quotes in verbatim environments
\IfFileExists{upquote.sty}{\usepackage{upquote}}{}
\IfFileExists{microtype.sty}{% use microtype if available
  \usepackage[]{microtype}
  \UseMicrotypeSet[protrusion]{basicmath} % disable protrusion for tt fonts
}{}
\makeatletter
\@ifundefined{KOMAClassName}{% if non-KOMA class
  \IfFileExists{parskip.sty}{%
    \usepackage{parskip}
  }{% else
    \setlength{\parindent}{0pt}
    \setlength{\parskip}{6pt plus 2pt minus 1pt}}
}{% if KOMA class
  \KOMAoptions{parskip=half}}
\makeatother
\usepackage{xcolor}
\setlength{\emergencystretch}{3em} % prevent overfull lines
\setcounter{secnumdepth}{5}
% Make \paragraph and \subparagraph free-standing
\ifx\paragraph\undefined\else
  \let\oldparagraph\paragraph
  \renewcommand{\paragraph}[1]{\oldparagraph{#1}\mbox{}}
\fi
\ifx\subparagraph\undefined\else
  \let\oldsubparagraph\subparagraph
  \renewcommand{\subparagraph}[1]{\oldsubparagraph{#1}\mbox{}}
\fi


\providecommand{\tightlist}{%
  \setlength{\itemsep}{0pt}\setlength{\parskip}{0pt}}\usepackage{longtable,booktabs,array}
\usepackage{calc} % for calculating minipage widths
% Correct order of tables after \paragraph or \subparagraph
\usepackage{etoolbox}
\makeatletter
\patchcmd\longtable{\par}{\if@noskipsec\mbox{}\fi\par}{}{}
\makeatother
% Allow footnotes in longtable head/foot
\IfFileExists{footnotehyper.sty}{\usepackage{footnotehyper}}{\usepackage{footnote}}
\makesavenoteenv{longtable}
\usepackage{graphicx}
\makeatletter
\def\maxwidth{\ifdim\Gin@nat@width>\linewidth\linewidth\else\Gin@nat@width\fi}
\def\maxheight{\ifdim\Gin@nat@height>\textheight\textheight\else\Gin@nat@height\fi}
\makeatother
% Scale images if necessary, so that they will not overflow the page
% margins by default, and it is still possible to overwrite the defaults
% using explicit options in \includegraphics[width, height, ...]{}
\setkeys{Gin}{width=\maxwidth,height=\maxheight,keepaspectratio}
% Set default figure placement to htbp
\makeatletter
\def\fps@figure{htbp}
\makeatother

\makeatletter
\@ifpackageloaded{caption}{}{\usepackage{caption}}
\AtBeginDocument{%
\ifdefined\contentsname
  \renewcommand*\contentsname{Table of contents}
\else
  \newcommand\contentsname{Table of contents}
\fi
\ifdefined\listfigurename
  \renewcommand*\listfigurename{List of Figures}
\else
  \newcommand\listfigurename{List of Figures}
\fi
\ifdefined\listtablename
  \renewcommand*\listtablename{List of Tables}
\else
  \newcommand\listtablename{List of Tables}
\fi
\ifdefined\figurename
  \renewcommand*\figurename{Figure}
\else
  \newcommand\figurename{Figure}
\fi
\ifdefined\tablename
  \renewcommand*\tablename{Table}
\else
  \newcommand\tablename{Table}
\fi
}
\@ifpackageloaded{float}{}{\usepackage{float}}
\floatstyle{ruled}
\@ifundefined{c@chapter}{\newfloat{codelisting}{h}{lop}}{\newfloat{codelisting}{h}{lop}[chapter]}
\floatname{codelisting}{Listing}
\newcommand*\listoflistings{\listof{codelisting}{List of Listings}}
\makeatother
\makeatletter
\makeatother
\makeatletter
\@ifpackageloaded{caption}{}{\usepackage{caption}}
\@ifpackageloaded{subcaption}{}{\usepackage{subcaption}}
\makeatother
\ifLuaTeX
  \usepackage{selnolig}  % disable illegal ligatures
\fi
\usepackage{bookmark}

\IfFileExists{xurl.sty}{\usepackage{xurl}}{} % add URL line breaks if available
\urlstyle{same} % disable monospaced font for URLs
\hypersetup{
  pdftitle={Bonus Credit Assignment},
  pdfauthor={Hans Martinez},
  colorlinks=true,
  linkcolor={blue},
  filecolor={Maroon},
  citecolor={Blue},
  urlcolor={Blue},
  pdfcreator={LaTeX via pandoc}}

\title{Bonus Credit Assignment}
\usepackage{etoolbox}
\makeatletter
\providecommand{\subtitle}[1]{% add subtitle to \maketitle
  \apptocmd{\@title}{\par {\large #1 \par}}{}{}
}
\makeatother
\subtitle{Solution}
\author{Hans Martinez}
\date{2024-02-11}

\begin{document}
\maketitle

\section*{Upstream and Downstream
Monopolists}\label{upstream-and-downstream-monopolists}
\addcontentsline{toc}{section}{Upstream and Downstream Monopolists}

One interesting market structure is where a monopoly (\emph{upstream
monopolist}) produces output that is used as a factor of production by
another monopolist (\emph{downstream monopolist}). Suppose that the
upstream monopolist produces output \(M\) at a constant marginal cost
\(c\). The upstream monopolist sells the \(M\)-factor to the downstream
monopolist at a price \(k\). The downstream monopolist uses the
\(M\)-factor to produce output \(Q\) according to its production
function \(Q(M)=eM\). The output of the downstream monopolist is sold in
a consumer market whose inverse demand curve is linear \(P(Q)=a-bQ\).
Assume that the downstream monopolist has no other cost of production
other than the unit price \(q\) that it must pay to the upstream
monopolist.

\begin{enumerate}
\def\labelenumi{\alph{enumi}.}
\tightlist
\item
  Set the maximization problem of the downstream monopolist in terms of
  the input, \(M\). You can do this. Think about what's the total
  revenue of the downstream monopolist and what the total cost is for
  producing each unit of output.
\end{enumerate}

\[
\begin{aligned}     
\max \pi &= P(Q(M))Q(M)-kM \\
        & = (a-b(eM))eM-kM
\end{aligned}
\]

\begin{enumerate}
\def\labelenumi{\alph{enumi}.}
\setcounter{enumi}{1}
\tightlist
\item
  Solve the profit-maximization problem of the downstream monopolist and
  find the optimal input in terms of the parameters, \(M(a,b,e,k)\).
\end{enumerate}

\[
\begin{aligned}
    (a-beM)e+eM(-be)-k&=0 \\
    ae-be^2M-be^2M-k&=0 \\
    M &=\frac{ae-k}{2be^2}
\end{aligned}
\]

\begin{enumerate}
\def\labelenumi{\alph{enumi}.}
\setcounter{enumi}{2}
\tightlist
\item
  The optimal input equation, \(M(a,b,e,k)\), determines the factor
  demand function for the upstream monopolist. This function tells us
  the relationship between the factor price the upstream monopolist sets
  and the amount of the factor that the downstream monopolist will
  demand from the upstream monopolist. So, solve for the \(k\), to get
  the inverse demand curve of the \(M\)-factor.
\end{enumerate}

\[
k=ae-2be^2M
\]

\begin{enumerate}
\def\labelenumi{\alph{enumi}.}
\setcounter{enumi}{3}
\tightlist
\item
  Calculate the marginal revenue associated with the \(M\)-factor
  inverse demand and set it equal to the marginal cost of the upstream
  monopolist. Solve for \(M\). This is the optimal quantity that the
  upstream monopolist will produce of the \(M\)-factor.
\end{enumerate}

\[
\begin{aligned}
    ae-4be^2M &= c \\
    M &= \frac{ae-c}{4be^2}
\end{aligned}
\]

\begin{enumerate}
\def\labelenumi{\alph{enumi}.}
\setcounter{enumi}{4}
\tightlist
\item
  Because the upstream monopolist will only produce the optimal
  quantity, the downstream monopolist can only buy this amount of the
  \(M\)-factor. Find the output the downstream monopolist will produce
  and the price it will charge in the consumer market. You should have
  an equation of downstream monopolist output, \(Q(a,b,e,c)\), and
  price, \(P(a,e,c)\).
\end{enumerate}

\[
\begin{aligned}
    Q=eM=e\left(\frac{ae-c}{4be^2}\right)&=\frac{ae-c}{4be} \\
    P=a-bQ=a-b\left(\frac{ae-c}{4be}\right)&=\frac{3ae+c}{4e}
\end{aligned}
\]

\begin{enumerate}
\def\labelenumi{\alph{enumi}.}
\setcounter{enumi}{5}
\tightlist
\item
  Let's compare this output to the amount that would be produced by a
  single integrated monopolist. To do this, suppose the upstream and
  downstream firms merged into an \emph{integrated} monopolist. The
  integrated monopolist would face the inverse demand function,
  \(P(Q)=a-bQ\), and the production function of the downstream
  monopolist \(Q=eM\), and the constant marginal cost of the upstream
  monopolist, \(c\). Find the profit-maximizing output and price of the
  integrated monopolist.
\end{enumerate}

The profit-maximization problem for the integrated monopolist is \[
\max_M \pi = (a-b(eM))(eM)-cM
\]

Then, the optimal input quantity demanded is \[
\begin{aligned}
    (a-beM)e+eM(-be)-c &=0\\
    ae-2be^2M-c&=0\\
    M=\frac{ae-c}{2be^2}
\end{aligned}
\]

This level of input demand implies the following output and price

\[
\begin{aligned}
    Q=eM=e\left(\frac{ae-c}{2be^2}\right)&=\frac{ae-c}{2be} \\
    P=a-bQ=a-b\left(\frac{ae-c}{2be}\right)&=\frac{ae+c}{2e}
\end{aligned}
\]

\begin{enumerate}
\def\labelenumi{\alph{enumi}.}
\setcounter{enumi}{6}
\tightlist
\item
  Finally, compare the outputs of the integrated with the non-integrated
  monopolist. What condition must be true in order for the price of the
  non-integrated market (downstream and upstream monopolists) to be
  greater than the price of the integrated monopolist?
\end{enumerate}

Let's call \(Q^I\) the output of the integrated monopolist, and \(Q^N\)
the output where the upstream and downstream monopolists are not
integrated.

\[
\begin{aligned}
    \frac{Q^I}{Q^N} &= Q^I (Q^N)^{-1}\\
                    &=\frac{ae-c}{2be}\left(\frac{ae-c}{4be}\right)^{-1} \\
                    &= \frac{ae-c}{2be}\frac{4be}{ae-c} \\
                    &= 2
\end{aligned}
\]

The integrated monopolist produces twice the output of the
upstream-downstream pair monopolists. This happens because of the double
markup. The upstream monopolist raises its price above its marginal cost
and then the downstream monopolist raises its price above this already
marked-up cost.

We expect that, everything else held constant, the output price of the
non-integrated monopolists must be greater than the output price of the
integrated one. In order for prices to make sense, the following
condition must be true.

\[
\begin{aligned}
    P^N &> P^I \\
    \frac{3ae+c}{4e}&>\frac{ae+c}{2e}\\
    2e(3ae+c)&>4e(ae+c)\\
    3ae+c &> 2ae+2c \\
    ae &> c
\end{aligned}
\]



\end{document}
