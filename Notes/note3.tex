% Options for packages loaded elsewhere
\PassOptionsToPackage{unicode}{hyperref}
\PassOptionsToPackage{hyphens}{url}
\PassOptionsToPackage{dvipsnames,svgnames,x11names}{xcolor}
%
\documentclass[
]{article}

\usepackage{amsmath,amssymb}
\usepackage{iftex}
\ifPDFTeX
  \usepackage[T1]{fontenc}
  \usepackage[utf8]{inputenc}
  \usepackage{textcomp} % provide euro and other symbols
\else % if luatex or xetex
  \usepackage{unicode-math}
  \defaultfontfeatures{Scale=MatchLowercase}
  \defaultfontfeatures[\rmfamily]{Ligatures=TeX,Scale=1}
\fi
\usepackage{lmodern}
\ifPDFTeX\else  
    % xetex/luatex font selection
\fi
% Use upquote if available, for straight quotes in verbatim environments
\IfFileExists{upquote.sty}{\usepackage{upquote}}{}
\IfFileExists{microtype.sty}{% use microtype if available
  \usepackage[]{microtype}
  \UseMicrotypeSet[protrusion]{basicmath} % disable protrusion for tt fonts
}{}
\makeatletter
\@ifundefined{KOMAClassName}{% if non-KOMA class
  \IfFileExists{parskip.sty}{%
    \usepackage{parskip}
  }{% else
    \setlength{\parindent}{0pt}
    \setlength{\parskip}{6pt plus 2pt minus 1pt}}
}{% if KOMA class
  \KOMAoptions{parskip=half}}
\makeatother
\usepackage{xcolor}
\setlength{\emergencystretch}{3em} % prevent overfull lines
\setcounter{secnumdepth}{5}
% Make \paragraph and \subparagraph free-standing
\ifx\paragraph\undefined\else
  \let\oldparagraph\paragraph
  \renewcommand{\paragraph}[1]{\oldparagraph{#1}\mbox{}}
\fi
\ifx\subparagraph\undefined\else
  \let\oldsubparagraph\subparagraph
  \renewcommand{\subparagraph}[1]{\oldsubparagraph{#1}\mbox{}}
\fi


\providecommand{\tightlist}{%
  \setlength{\itemsep}{0pt}\setlength{\parskip}{0pt}}\usepackage{longtable,booktabs,array}
\usepackage{calc} % for calculating minipage widths
% Correct order of tables after \paragraph or \subparagraph
\usepackage{etoolbox}
\makeatletter
\patchcmd\longtable{\par}{\if@noskipsec\mbox{}\fi\par}{}{}
\makeatother
% Allow footnotes in longtable head/foot
\IfFileExists{footnotehyper.sty}{\usepackage{footnotehyper}}{\usepackage{footnote}}
\makesavenoteenv{longtable}
\usepackage{graphicx}
\makeatletter
\def\maxwidth{\ifdim\Gin@nat@width>\linewidth\linewidth\else\Gin@nat@width\fi}
\def\maxheight{\ifdim\Gin@nat@height>\textheight\textheight\else\Gin@nat@height\fi}
\makeatother
% Scale images if necessary, so that they will not overflow the page
% margins by default, and it is still possible to overwrite the defaults
% using explicit options in \includegraphics[width, height, ...]{}
\setkeys{Gin}{width=\maxwidth,height=\maxheight,keepaspectratio}
% Set default figure placement to htbp
\makeatletter
\def\fps@figure{htbp}
\makeatother
% definitions for citeproc citations
\NewDocumentCommand\citeproctext{}{}
\NewDocumentCommand\citeproc{mm}{%
  \begingroup\def\citeproctext{#2}\cite{#1}\endgroup}
\makeatletter
 % allow citations to break across lines
 \let\@cite@ofmt\@firstofone
 % avoid brackets around text for \cite:
 \def\@biblabel#1{}
 \def\@cite#1#2{{#1\if@tempswa , #2\fi}}
\makeatother
\newlength{\cslhangindent}
\setlength{\cslhangindent}{1.5em}
\newlength{\csllabelwidth}
\setlength{\csllabelwidth}{3em}
\newenvironment{CSLReferences}[2] % #1 hanging-indent, #2 entry-spacing
 {\begin{list}{}{%
  \setlength{\itemindent}{0pt}
  \setlength{\leftmargin}{0pt}
  \setlength{\parsep}{0pt}
  % turn on hanging indent if param 1 is 1
  \ifodd #1
   \setlength{\leftmargin}{\cslhangindent}
   \setlength{\itemindent}{-1\cslhangindent}
  \fi
  % set entry spacing
  \setlength{\itemsep}{#2\baselineskip}}}
 {\end{list}}
\usepackage{calc}
\newcommand{\CSLBlock}[1]{\hfill\break\parbox[t]{\linewidth}{\strut\ignorespaces#1\strut}}
\newcommand{\CSLLeftMargin}[1]{\parbox[t]{\csllabelwidth}{\strut#1\strut}}
\newcommand{\CSLRightInline}[1]{\parbox[t]{\linewidth - \csllabelwidth}{\strut#1\strut}}
\newcommand{\CSLIndent}[1]{\hspace{\cslhangindent}#1}

\makeatletter
\@ifpackageloaded{caption}{}{\usepackage{caption}}
\AtBeginDocument{%
\ifdefined\contentsname
  \renewcommand*\contentsname{Table of contents}
\else
  \newcommand\contentsname{Table of contents}
\fi
\ifdefined\listfigurename
  \renewcommand*\listfigurename{List of Figures}
\else
  \newcommand\listfigurename{List of Figures}
\fi
\ifdefined\listtablename
  \renewcommand*\listtablename{List of Tables}
\else
  \newcommand\listtablename{List of Tables}
\fi
\ifdefined\figurename
  \renewcommand*\figurename{Figure}
\else
  \newcommand\figurename{Figure}
\fi
\ifdefined\tablename
  \renewcommand*\tablename{Table}
\else
  \newcommand\tablename{Table}
\fi
}
\@ifpackageloaded{float}{}{\usepackage{float}}
\floatstyle{ruled}
\@ifundefined{c@chapter}{\newfloat{codelisting}{h}{lop}}{\newfloat{codelisting}{h}{lop}[chapter]}
\floatname{codelisting}{Listing}
\newcommand*\listoflistings{\listof{codelisting}{List of Listings}}
\makeatother
\makeatletter
\makeatother
\makeatletter
\@ifpackageloaded{caption}{}{\usepackage{caption}}
\@ifpackageloaded{subcaption}{}{\usepackage{subcaption}}
\makeatother
\ifLuaTeX
  \usepackage{selnolig}  % disable illegal ligatures
\fi
\usepackage{bookmark}

\IfFileExists{xurl.sty}{\usepackage{xurl}}{} % add URL line breaks if available
\urlstyle{same} % disable monospaced font for URLs
\hypersetup{
  pdftitle={Note 3},
  pdfauthor={Hans Martinez},
  colorlinks=true,
  linkcolor={blue},
  filecolor={Maroon},
  citecolor={Blue},
  urlcolor={Blue},
  pdfcreator={LaTeX via pandoc}}

\title{Note 3}
\author{Hans Martinez}
\date{2024-01-30}

\begin{document}
\maketitle

\section[The Impact of Taxes on a Monopolist]{\texorpdfstring{The Impact
of Taxes on a
Monopolist\footnote{Examples expanded from Varian (2014) p.~464.}}{The Impact of Taxes on a Monopolist}}\label{the-impact-of-taxes-on-a-monopolisttaken}

Consider a firm with constant marginal costs, \(MC=c\), that faces a
linear demand, \(P=a-bQ\). What happens to the price charged when a
quantity tax \(t\) is imposed?

The marginal cost goes up by the amount of the tax, \(MC=c+t\). The
intersection of marginal revenue, \(MR=a-2bQ\), and marginal cost moves
to the left.

Algebraically, the optimality condition becomes now \[
a-2bQ=c+t
\]

Solving for Q,

\begin{equation}\phantomsection\label{eq-opt-q}{
Q=\frac{a-c-t}{2b}
}\end{equation}

Thus, the change in output due to a change in the quantity tax is

\[
\frac{dQ}{dt}=-\frac{1}{2b}
\]

Plugging Equation~\ref{eq-opt-q} into the demand curve

\begin{equation}\phantomsection\label{eq-opt-p}{
P=\frac{a+c+t}{2}
}\end{equation}

Taking the derivative of Equation~\ref{eq-opt-p} with respect \(t\), we
get

\[
\frac{dP}{dt}=\frac{1}{2}
\]

Since the demand curve is half as steep as the marginal revenue curve,
the price goes up by half the amount of the tax. This happens because of
the assumptions of the linear demand curve and constant marginal costs.
These assumptions imply that the price will increase by \emph{less than
the tax increase}. Is this likely to be true in general?

Consider now the case of a monopolist facing a constant-elasticity
demand curve, \(P=aQ^{-b}\). The price elasticity of demand is

\[
\begin{aligned}
\epsilon_{Q,P}&=\frac{dQ}{dP}\frac{P}{q}\\
    &=-baP^{-b-1}\frac{P}{aQ^{-b}}\\
    &=-b
\end{aligned}
\]

Then, the marginal revenue in terms of the price elasticity of demand
now becomes

\[
\begin{aligned}
MR&=P\left(1-\frac{1}{ |\epsilon_{Q,P}| }\right)\\
    &=P\left(1-\frac{1}{ |b| }\right)
\end{aligned}
\]

Next, using the optimality condition and solving for price,

\[
\begin{aligned}
P\left(1-\frac{1}{ |\epsilon_{Q,P}| }\right)&=c+t\\
    P&=\frac{c+t}{\left(1-\frac{1}{ |b| }\right)}
\end{aligned}
\]

so that, \begin{equation}\phantomsection\label{eq-dpdt-e}{
\frac{dP}{dt}=\frac{1}{\left(1-\frac{1}{ |b| }\right)}
}\end{equation}

Because the monopolist will never operate in the inelastic part of the
demand, \(|b| >1\), then \(\frac{dP}{dt}> 1\). Equation~\ref{eq-dpdt-e}
says that, in this case, the monopolist will increase the price
\emph{more than the amount of the tax}.

Another kind of tax we can consider is the profit tax. In this case, the
monopolist is required to pay a fraction \(\tau\) of its profits, to the
government. In addition, the monopolist is only allowed to deduct a
fraction \(\lambda\le1\) of its total costs from total revenues.

The maximization problem that the monopolist faces is then

\begin{equation}\phantomsection\label{eq-profit-tax}{
\max_{Q} (1-\tau)[TR(Q)-\lambda TC(Q)] - (1-\lambda) TC(Q)
}\end{equation}

Taking the derivative with respect to \(Q\), the optimality condition
becomes

\begin{equation}\phantomsection\label{eq-foc-profit-tax}{
\frac{dTR}{dQ} = \frac{dTC}{dQ}\left(\lambda + \frac{1-\lambda}{1-\tau}\right)
}\end{equation}

Equation~\ref{eq-foc-profit-tax} looks familiar. Our well-known
optimality condition now contains a \emph{wedge},
\(\left(\lambda + \frac{1-\lambda}{1-\tau}\right)\), multiplying the
marginal cost.

Suppose the demand is linear, \(P=a-bQ\), and the marginal costs are
constant, \(\frac{dTC}{dQ}=c\), then the optimal quantity and price
become

\[
Q^*=\frac{a-c\left(\lambda + \frac{1-\lambda}{1-\tau}\right)}{2b}
\]

\[
P^*=\frac{a+c\left(\lambda + \frac{1-\lambda}{1-\tau}\right)}{2b}
\]

Now the optimal decision of the firm depends on the profit tax \(\tau\)
and the fraction of the costs the firm is allowed to deduct \(\lambda\).

If \(\lambda=1\), then the optimal price and quantities do not change
from the case where there is no profit tax. In other words, if the firm
is allowed to deduct all of its costs from total revenue for tax
purposes, then the same choice of quantity and price that maximizes its
profits under no profit tax also maximizes its profits under such a tax.

If \(\lambda=0\), the marginal cost wedge becomes
\(\frac{1}{(1-\tau)}\). Since \(\tau<1\), then \(\frac{1}{(1-\tau)}>1\).
This means that the optimal quantity will decrease and the optimal price
will increase. In this case, \[
\frac{dP}{d\tau}=\frac{c}{2b}\frac{1}{(1-\tau)^2} > 0
\]

In other words, when the firm cannot deduct any percentage from the
profit tax, an increase of one percentage point in the profit tax will
increase the output price by
\(\frac{dP}{d\tau}=\frac{c}{2b}\frac{1}{(1-\tau)^2}\).

Another peculiar case is where \(\lambda=\tau\). In this case, the
marginal cost wedge becomes \((\lambda+1)>1\) (or \(\tau+1\), since they
are equal). So, the firm's marginal cost increases by \((\lambda+1)\).
Thus,

\[
\frac{dP}{d\tau}=\frac{dP}{d\lambda}=\frac{c}{2}>0
\]

For a one percentage point increase of the profit tax (or the fraction
of the deductible cost), the monopolist will increase the price by half
the marginal cost.

If \(\lambda\not=\tau\), the marginal cost increases by
\(\left(\lambda + \frac{1-\lambda}{1-\tau}\right)>1\).

So we have,

\begin{equation}\phantomsection\label{eq-dpdt-profit}{
\begin{aligned}
\frac{dP}{d\lambda} &= -\frac{c\tau}{2(1-\tau)} < 0\\
\frac{dP}{d\tau} &= \frac{c(1-\lambda)}{2(1-\tau)^2} > 0
\end{aligned}
}\end{equation}

These equations indicate that, holding the profit tax constant, the
price will decrease as the firm is allowed to deduct a greater fraction
of its total costs. Likewise, holding the deductible fraction of total
costs constant, a greater profit tax will increase the price.

\section*{References}\label{references}
\addcontentsline{toc}{section}{References}

\phantomsection\label{refs}
\begin{CSLReferences}{1}{0}
\bibitem[\citeproctext]{ref-Varian2014}
Varian, Hal R. 2014. \emph{Intermediate Microeconomics with Calculus}.
NY: W.W. Norton; Company.

\end{CSLReferences}



\end{document}
